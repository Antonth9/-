\documentclass[a4paper,12pt]{article}
\usepackage[utf8]{inputenc}
\usepackage[russian]{babel}
\usepackage{tocloft}
\usepackage{indentfirst} % Пакет для отступа первого абзаца
\usepackage[left=20mm, top=25mm, right=20mm, bottom=20mm, head= 20mm, nofoot]{geometry}
\setlength{\parindent}{1cm}

\title{Название документа}
\author{Автор}
\date{\today}

\begin{document}

\maketitle

\tableofcontents % Генерация содержания
\newpage
\section{Введение}
\newpage

\section{Концепция}
\subsection{Введение}

Игрок выступает в роли члена экипажа космического корабля, потерпевшего крушение на неизведанной планете, где он должен вылечиться, восстановить память и найти способ покинуть опасное место, полное тайн и загадочных существ. Игра сочетает элементы экшена, стратегии и RPG, ориентирована на широкую аудиторию, включая как казуальных, так и хардкорных игроков.
\subsection{Жанр и аудитория}
\begin{itemize}
    \item Жанр: Экшен
    \item Возрастная группа: 16-35 лет
    \item Игра нацелена на любителей экшена и прюклечений, а также фанатов рогаликов и шутеров.
\end{itemize}
\subsection{Основные особенности игры}
\begin{itemize}
    \item Генерация нового окружения при каждой перезагрузке, создающая уникальные испытания.
    \item Возможность улучшения и модификации оружия и способностей.
    \item Элементы кооперативного режима для совместных сражений с боссами.
    \item Примерный объем игры: 15-20 часов основного прохождения, в основе времени лежит набор игроком опыта и знаний, необходимых для успешного выживания и прохождения без смерти.
\end{itemize}
\subsection{Описание игры}
Игрок начинает с травмами и потерей памяти, исследуя неизведанную планету, полную опасных существ и тайн. Он собирает ресурсы, создаёт оружие, броню и медикаменты, а также решает головоломки, чтобы получить информацию о своем прошлом и о планете. Игровой процесс включает битвы с врагами, прокачку персонажа и взаимодействие с окружающей средой, создавая атмосферу неизвестности и опасности.
\subsection{Предпосылки создания}
\begin{itemize}
    \item Общие тенденции рынка: Растущий интерес к играм с элементами выживания, исследованием открытых миров и кооперативным геймплеем. Популярность рогаликов и шутеров также подчеркивает актуальность данного проекта.
    \item Вопросы, связанные с лицензированием: Игра не требует лицензирования, так как использует оригинальные механики и дизайн.
\end{itemize}
\subsection{Платформа}

    \noindentПлатформа: ПК.
    Минимальные системные требования :
\begin{itemize}
    \item Операционная система: Windows 10
    \item Процессор: Intel Core i3 10 поколения или аналогичный
    \item Оперативная память: 8 ГБ
    \item Видеокарта: NVIDIA GeForce GTX 660 или аналогичная
    \item Место на диске: 10 ГБ
\end{itemize}
    \noindentРекомендуемые системные требования:
\begin{itemize}
    \item Операционная система: Windows 11
    \item Процессор: Intel Core i5 10 поколения или аналогичный
    \item Оперативная память: 16 ГБ
    \item Видеокарта: NVIDIA GeForce GTX 970 или аналогичная
    \item Место на диске: 15 ГБ
\end{itemize}
\noindentДополнительное оборудование: Не требуется.
\newpage

\section{Функциональная спецификация}
\subsection{Принципы игры}
\subsubsection{Суть игрового процесса}
{\textbf{Корабль как передвижная база}}

Игрок начинает с поврежденным космическим кораблем, который служит не только средством передвижения, но и базой для хранения ресурсов и создания предметов. На первой локации игрок получает возможность отремонтировать корабль, используя собранные материалы.

Функции корабля:
Хранилище: Игрок может хранить собранные ресурсы, оружие и предметы, что позволяет ему управлять инвентарем и планировать свои действия.
Создание: Корабль будет оснащен станциями для крафта, где игрок сможет создавать оружие, броню и другие полезные предметы, используя найденные ресурсы.
Перемещение: После ремонта корабль открывает доступ к новым локациям, что позволяет игроку исследовать различные экосистемы и находить уникальные ресурсы.

 
\textbf{{Опасное окружение}}

Игрок попадает в мир, наполненный опасностями и неизвестными существами. Окружение является неотъемлемой частью игрового процесса, создавая атмосферу напряженности и непредсказуемости.

Монстры:
Разнообразные виды монстров, которые могут скрываться в густой растительности или неожиданно нападать на игрока. Некоторые из них могут быть агрессивными, другие — осторожными.
Игрок должен быть внимательным и использовать окружающую среду для своей выгоды, например, прятаться за укрытиями или использовать растительность для маскировки.
Экологические опасности:
Игрок сталкивается с природными препятствиями, такими как ядовитые растения, опасные ямы и другие ловушки, которые могут повредить его здоровье или затруднить передвижение.

\textbf{Возможность создания оружия и брони}

Создание оружия и брони является ключевым элементом для выживания в опасном мире. Игроку предоставляется возможность собирать материалы и создавать более мощные варианты вооружения и защиты.
Игрок может создавать различные виды оружия, от простых до сложных, используя ресурсы, найденные в мире. Более сильные варианты вооружения увеличивают шансы на выживание в столкновениях с монстрами.
Броня позволяет уменьшить получаемый урон и дает дополнительные способности, такие как увеличение скорости или улучшенное восстановление здоровья.
С созданием мощного оружия и брони у игрока появляется возможность более активно взаимодействовать с монстрами, что позволяет ему не только избегать столкновений, но и атаковать врагов.

\textbf{Прокачка навыков персонажа}

Игроки могут развивать навыки своего персонажа, что добавляет элемент RPG в геймплей. Чем больше игрок использует определенные действия, тем более умелым он становится в них.
При частом использовании штурмовой винтовки, персонаж начинает легче контролировать отдачу, улучшая точность стрельбы и скорость перезарядки.
Навыки могут быть разделены на категории, такие как стрельба, ближний бой, крафтинг и выживание. Каждый навык будет развиваться на основе частоты использования, что делает игру более персонализированной.
Способности:
Игрок может открывать новые способности по мере повышения уровней навыков, что позволяет экспериментировать с различными стилями игры и подходами к бою.

\textbf{Нет права на ошибку}

Игровая механика построена на высокой степени сложности, где каждое неверное решение может привести к гибели персонажа.
В случае смерти игрока игра начинается заново, что создает напряжение и заставляет тщательно продумывать каждое действие.
Игроку придется учитывать все риски, взвешивать свои действия и принимать стратегические решения в бою, чтобы избежать фатальных ошибок.
Стратегия и планирование:
Игроку необходимо заранее планировать свои действия, используя ресурсы и навыки, чтобы минимизировать риски. Это создает захватывающий и напряженный игровой процесс, где каждое решение имеет значение.
\subsubsection{Ход игры и сюжет}
\begin{center}
 \textbf{Обычная игровая сессия} 
\end{center}
\textbf{Начало сессии}

Игрок запускает игру и загружается в свой поврежденный космический корабль, который находится на первой локации — густой тропический лес. На экране есть миникарта, но на ней есть только пройденные зоны, поэтому игроку придётся столкнуться с пугающей неизвестностью этого мира.
Исследование локации
Игрок покидает корабль, вооружившись базовым оружием и броней. Он начинает исследовать окрестности, внимательно осматривая растительность и подбирая ресурсы. В процессе игры он сталкивается с внезапными событиями. Например: внезапно из-за кустов выскакивает монстр — агрессивный зверь. Игрок быстро прячется за укрытием, чтобы избежать атаки, или вступает в бой.

\textbf{Столкновение с врагом}

Игрок решает атаковать монстра, используя свое оружие. Он делает несколько ударов, но промахивается — враг наносит ему урон. Игрок теряет здоровье, и осознав риски снова вступает в бой. После напряженной схватки ему удается победить монстра, и он получает ценные материалы, которые можно использовать для улучшения своего снаряжения. Или он умрёт и получит информацию о способностях начального противника, возможно сможет придумать тактику для победы в следующем столкновении.

\textbf{Крафтинг и улучшение}

Вернувшись на корабль, игрок открывает меню крафта. Он использует собранные материалы для создания нового оружия и улучшения брони. Теперь у него есть более мощный лазерный пистолет и защитный жилет, что значительно увеличивает его шансы на выживание.

\textbf{Дальнейшее исследование}

С новыми предметами игрок решает исследовать соседнюю локацию, отмеченную на карте. Он подходит к входу в пещеру, где, по слухам, можно найти редкие ресурсы. Вход в пещеру темный и угрожающий, но игрок готов рискнуть. Угроза: Внутри пещеры он сталкивается с ловушками и новыми видами монстров, которые требуют от него быстрой реакции и стратегического мышления. Игрок использует снаряжение и свои новые навыки, чтобы преодолеть эти препятствия.

\textbf{Прокачка навыков}

Во время боя игрок активно использует свое оружие, что позволяет ему прокачать навык стрельбы. После нескольких успешных атак он замечает, что его точность и скорость перезарядки значительно улучшились. Это вдохновляет его на дальнейшие эксперименты с различными стилями боя.

\textbf{Завершение сессии}
После успешного исследования пещеры и сбора редких ресурсов игрок решает вернуться на корабль, чтобы сохранить свои достижения. Он сохраняет игру и выходит из сессии.

\textbf{Итог}

В этой игровой сессии игрок взаимодействует с окружающим миром, принимает стратегические решения, исследует новые локации, сражается с монстрами и улучшает своего персонажа. Каждый элемент геймплея — от крафта до прокачки навыков — создает увлекательный и напряженный опыт, который заставляет игрока возвращаться к игре снова и снова.

\textbf{Сюжет}

Игрок выступает в роли безымянного персонажа, члена экипажа космического корабля, который потерпел крушение на неизведанной планете, покрытой густыми лесами и океанами, а также полными опасностей пещерами. С самого начала игры персонаж приходит в себя в корабле, страдая от амнезии и физического истощения. Он обнаруживает, что он совсем один, а сам корабль частично поврежден. Игрок начинает исследовать окружающую местность, собирая ресурсы и восстанавливая здоровье. Во время первых попыток выбраться он сталкивается с агрессивными существами, что заставляет его использовать свои инстинкты и навыки выживания. В процессе исследования он находит снаряжение, оставленных другими людьми, которые намекают на странные эксперименты, проводимые на борту корабля. Постепенно он начинает находить тела предыдущих солдат, сильно похожих на него, которые были частью эксперимента по созданию суперсолдат. Он обнаруживает, что каждый из них был клонирован с помощью машины, находящейся в закрытом отсеке корабля. Персонаж понимает, что он не потерял память, а является результатом неудачного эксперимента. Каждый раз, когда он умирает, машина клонирования создает нового без воспоминаний о предыдущих жизнях. В процессе игры игрок сталкивается с моральными дилеммами: он должен решить, что делать с этой правдой. Он может продолжать сражаться с монстрами, как его предшественники, или попытаться остановить эксперимент, чтобы прекратить цикл клонирования. Его целью становится восстановление корабля и изменение машины клонирования, чтобы прекратить бесконечное создание солдат и выбраться с планеты живым.

\subsection{Физическая модель}
\subsubsection{Динамика движения персонажа}
•	Физика передвижения: Персонаж будет иметь реалистичное движение, включая ускорение, замедление и инерцию. Например, при беге он должен постепенно набирать скорость и также замедляться, когда останавливается.

•	Прыжки и падения: В игре будет физика прыжков, учитывающая силу тяжести, чтобы персонаж мог прыгать и падать с разных высот. Падение с большой высоты может повредить здоровье, а также влиять на скорость восстановления.
\subsubsection{Взаимодействие с окружающей средой}
•	Разрушаемость объектов: Некоторые элементы окружающей среды могут быть разрушены (например, деревья, камни, трава), что добавляет тактические возможности в бою и исследовании.

•	Физика предметов: Реалистичное поведение предметов, которые можно поднимать, бросать или использовать в бою. Например, если он бросает деревянный ящик, он будет ломаться о твёрдую поверхность, как в игре “half-life”.

•	Гравитация и трение: В игре будет учитываться влияние гравитации на объекты, а также трение между поверхностями (например, скользкие поверхности или песок), что влияет на скорость передвижения.

•	Реакция на столкновения: В игре определяются взаимодействия между игроком и окружающей средой, а также с врагами и предметами. В зависимости от силы столкновения (например, при ударе о стену или при падении) могут происходить различные реакции: урон персонажу, повреждение объектов, изменение траектории движения.
\subsubsection{Экологические факторы}
•	Погодные условия: Реалистичные погодные эффекты (дождь, ветер, снег), которые могут влиять на видимость, движение и взаимодействие с окружающей средой. Например, дождь может сделать поверхность скользкой, а ветер может отклонять снаряды.

•	Динамика воды: В игре будут реализованы водные объекты, включая течение, волны и возможность плавания. Персонаж может утонуть или быть смещен течением.
\subsubsection{Боевые механики}
•	Физика оружия: Реалистичное поведение оружия, включая отдачу, разброс пуль и время перезарядки. Например, при стрельбе из огнестрельного оружия персонаж будет испытывать отдачу и замедляться, его оружие будет уносить в сторону.

•	Ближний бой: Реалистичные анимации и физика для ближнего боя, включая возможность блокировки и уклонения от атак врагов. Удары будут иметь физическое воздействие на врагов (например, отбрасывание или оглушение).
\subsubsection{Персонаж}
•	Влияние на персонажа: Каждый новый игрок может иметь небольшие изменения в физических характеристиках, что может влиять на его скорость, силу и выносливость. Это обусловлено набором случайных начальных навыков.

•	Состояние игрока: В зависимости от физического и ментального состояния, персонаж может получать временные бустеры или штрафы к физическим характеристикам, что влияет на его способность выживать и сражаться. Например, при окружении врагами, персонаж будет видеть сниженную четкость изображения, но в то же время он станет немного быстрее сильнее. Это симуляция выброса адреналина в организме.
\subsubsection{Игровой мир}
•	Разнообразные биомы: В игре будут представлены разные типы окружения, такие как леса, пустыни, горы, моря, пещеры. Каждая локация может иметь свои уникальные механики и вызовы.

\subsection{Персонаж игрока}
Возраст: 15-35 лет (от этого параметра зависят начальные характеристики персонажа и бонус к получаемому опыту)

Внешность настраеваемая 

Раса: Генномодифицированный человек.


Начальные профессии:
1. Инженер
Технические навыки: Увеличенная эффективность в крафтинге и ремонте оборудования.
Сила: Средняя.
Интеллект: Высокий.
Особые способности:
Увеличение прочности оружия и брони.
Возможность создавать ловушки и механизмы для защиты.
Начальное оружие: Модифицированный пистолет или автомат.
2. Медик
Основные характеристики:
Мудрость: Высокая (улучшение лечения и поддержки).
Ловкость: Средняя.
Сила: Низкая.
Особые способности:
Ускоренное лечение себя и союзников.
Создание медицинских предметов (аптечек, стимуляторов).
Начальное оружие: Небольшой пистолет или шокер.
3. Разведчик
Основные характеристики:
Ловкость: Высокая (повышенная скорость и уклонение).
Скрытность: Высокая (уменьшение вероятности обнаружения).
Сила: Низкая.
Особые способности:
Увеличенная дальность обзора и возможность находить скрытые локации.
Способность устанавливать ловушки для врагов.
Начальное оружие: Лук или пистолет с глушителем.
4. Боец
Основные характеристики:
Сила: Высокая (увеличение урона в ближнем бою).
Выносливость: Высокая (увеличение здоровья и защиты).
Интеллект: Низкая.
Особые способности:
Увеличение урона от оружия ближнего боя.
Способность блокировать атаки врагов.
Начальное оружие: Меч или дубинка.
5. Ученый
Основные характеристики:
Интеллект: Очень высокий (улучшение навыков исследования и крафтинга).
Мудрость: Средняя.
Сила: Низкая.
Особые способности:
Увеличенная эффективность в исследовании и анализе окружающей среды.
Возможность создавать уникальные предметы и технологии.
Начальное оружие: Плазменный пистолет или энергетическое ружье.
6. Хакер
Основные характеристики:
Интеллект: Высокий (улучшение навыков взлома и взаимодействия с техникой).
Ловкость: Средняя.
Сила: Низкая.
Особые способности:
Возможность взламывать системы безопасности и управлять дронами.
Увеличение шанса на успешное избегание боевых столкновений.
Начальное оружие: Электрошокер или пистолет.
\subsection{Элементы игры}
В игре локации и оружие будут разделен на 3 уровня, в порядке прогресса. 
\subsubsection{Оружие}
Оружие 1 уровня:
Кувалда - орудие ближнего боя дробящего действия, подходит для разрушения объектов
Мачете - нож, подходит для прохождения через заросли, можно использовать как режущее оружие.

Винтовка со скользящим затвором. Аналог Springfield или Gewehr. Имеет магазин на 5 патрон, возможна установка оптического прицела и глушителя. Использует винтовочные боеприпасы.

Помповое ружьё - аналог существующего Remington 870. Вмещает 4 патрона, мощное оружие для ближних дистанций. Будет иметь 2 модификации:

Расширенный магазин на 8 патрон.

Выбор типа боеприпасов:

Дробь Используются для охоты на некрупную лесную дичь, водоплавающих птиц и мелкого зверя.
Картечные патроны. Основной вид боеприпасов. Обычно идут для охоты на некрупного зверя.
Пулевые патроны. Их берут для охоты на зверя покрупнее. Применяются пули разных форм и видов – круглые, стрелочные турбинные и другие.

Штурмовая винтовка. Похожее на StG 44 или автомат калашникова. Вмещает 30 патрон, использует винтовочные боеприпасы. Можно установить ночной прицел и глушитель.

Пистолет. Аналог M1911. Вмещает 8 патрон малого калибра. Подойдет как начальное оружие или как средство для запугивания животных.

\textbf{Оружие 2 уровня:}

Топор - современный топор 21века. 

Современный автомат.
Марксманская винтовка.
Автоматический дробовик.

1. Ствол
Дешевые варианты:
Материалы: Сталь, алюминий.
Виды: Стандартный гладкий ствол, ствол с нарезами для увеличенной точности.
Дорогие варианты:
Материалы: Нержавеющая сталь, титановый сплав, композитные материалы (например, углеродное волокно).
Виды: Ствол с улучшенной баллистикой, ствол с интегрированным дульным тормозом.
2. Цевье (Handguard)
Дешевые варианты:

Материалы: Пластик, алюминий.
Виды: Простое прямое цевье, с базовыми текстурированными вставками.
Дорогие варианты:

Материалы: Карбоновые волокна, магний.
Виды: Модульные системы с возможностью установки аксессуаров, интегрированные системы охлаждения.
3. Приклад
Дешевые варианты:

Материалы: Пластик, дерево.
Виды: Фиксированный приклад, стандартный телескопический приклад.
Дорогие варианты:

Материалы: Углеродное волокно, алюминий.
Виды: Регулируемый приклад с интегрированными системами амортизации.
4. Пистолетная рукоятка
Дешевые варианты:

Материалы: Пластик, резина.
Виды: Стандартная рукоятка, рукоятка с базовой текстурой.
Дорогие варианты:

Материалы: Композиты, алюминий.
Виды: Эргономичные рукоятки с возможностью настройки под размер руки, рукоятки с интегрированными системами управления.
5. Спусковой механизм (Trigger)
Дешевые варианты:

Материалы: Сталь, алюминий.
Виды: Стандартный спусковой механизм.
Дорогие варианты:

Материалы: Нержавеющая сталь, специализированные сплавы.
Виды: Регулируемые спусковые механизмы с изменяемым усилием и ходом.
6. Оптика и прицельные устройства
Дешевые варианты:

Материалы: Пластик, стекло.
Виды: Простые коллиматорные прицелы, механические прицельные устройства.
Дорогие варианты:

Материалы: Алюминий, высококачественное стекло.
Виды: Цифровые прицелы с возможностью ночного видения, тепловизионные прицелы.
7. Планки Picatinny или M-LOK
Дешевые варианты:

Материалы: Сталь, алюминий.
Виды: Стандартные планки без дополнительных функций.
Дорогие варианты:

Материалы: Углеродное волокно, магний.
Виды: Легкие модульные системы с интеграцией для различных аксессуаров.
8. Дульный тормоз и компенсаторы
Дешевые варианты:

Материалы: Сталь.
Виды: Простые дульные тормоза.
Дорогие варианты:

Материалы: Титан, специализированные сплавы.
Виды: Комплексные дульные тормоза с минимизацией отдачи и улучшением точности.
9. Магазины
Дешевые варианты:

Материалы: Пластик, сталь.
Виды: Стандартные магазины с фиксированным количеством патронов.
Дорогие варианты:

Материалы: Композиты, алюминий.
Виды: Модульные магазины с возможностью быстрой замены, магазины с увеличенной емкостью.
10. Крепления для аксессуаров
Дешевые варианты:

Материалы: Пластик, сталь.
Виды: Стандартные крепления.
Дорогие варианты:

Материалы: Алюминий, магний.
Виды: Модульные системы с возможностью быстрой установки и снятия аксессуаров.
11. Кожух для затвора (Bolt Carrier Group)
Дешевые варианты:

Материалы: Сталь.
Виды: Стандартный кожух.
Дорогие варианты:

Материалы: Нержавеющая сталь, титановый сплав.
Виды: Усовершенствованные кожухи с улучшенной надежностью и долговечностью.
12. Затвор
Дешевые варианты:

Материалы: Сталь.
Виды: Стандартный затвор.
Дорогие варианты:
Материалы: Нержавеющая сталь, специализированные сплавы.
Виды: Затворы с улучшенной геометрией для повышения надежности и скорости перезарядки.

13. Дульные устройства

1. Дульные тормоза
Стандартные дульные тормоза: Устройства, которые уменьшают подъем ствола при выстреле, позволяя более точно вести огонь.
Многофункциональные дульные тормоза: Устройства, которые не только уменьшают отдачу, но и помогают контролировать боковые смещения.
2. Компенсаторы
Стандартные компенсаторы: Устройства, которые уменьшают как вертикальную, так и горизонтальную отдачу, улучшая контроль над оружием.
Регулируемые компенсаторы: Позволяют пользователю настраивать уровень компенсации в зависимости от типа боеприпасов или условий стрельбы.
3. Глушители
Стандартные глушители: Устройства, которые уменьшают звуковую волну, создаваемую при выстреле, позволяя вести стрельбу более скрытно.
Интегрированные глушители: Глушители, встроенные в ствол, которые обеспечивают более компактный дизайн и улучшенные характеристики.
4. Дульные насадки
Дульные насадки для стрельбы дробью: Устройства, которые позволяют использовать дробовые патроны, увеличивая универсальность оружия.
Насадки для специальных боеприпасов: Например, для стрельбы нелетальными патронами или патронами с расширяющейся пулей.
5. Экспансивные дульные устройства
Дульные устройства с взрывчатыми зарядами: Устройства, которые активируются при выстреле и создают взрыв, наносящий урон в зоне поражения.
Дульные устройства с зажигательными элементами: Устройства, которые воспламеняют боеприпасы, создавая огненные эффекты при выстреле.
6. Дульные устройства для повышения точности
Стволы с интегрированными дульными устройствами: Стволы, которые имеют встроенные устройства для улучшения аэродинамических характеристик пули.
Системы управления отдачей: Устройства, которые регулируют поток газов, уменьшая отдачу и улучшая стабильность при стрельбе.
7. Патронники с изменяемым дульным устройством
Модульные системы: Позволяют менять дульные устройства в зависимости от боеприпасов и тактики использования.
Системы быстрой замены: Дульные устройства, которые можно быстро заменить в полевых условиях для адаптации к различным ситуациям.

14. Боеприпасы

1. Стандартные патроны
5.56x45 мм NATO: Широко используемый патрон для штурмовых винтовок, обеспечивает хорошую точность и дальность.
7.62x51 мм NATO: Патрон для более мощных винтовок, с высокой пробивной способностью.
2. Патроны с повышенной проникающей способностью
AP (Armor-Piercing): Патроны с сердечником из твердых материалов (например, сталь или tungsten), предназначенные для пробивания бронежилетов и другой защиты.
API (Armor-Piercing Incendiary): Патроны, которые не только проникают в броню, но и воспламеняются при попадании.
3. Патроны с расширяющейся пулей
Hollow Point (HP): Патроны с полой пулей, которые расширяются при попадании, увеличивая урон и останавливающую силу.
Soft Point (SP): Патроны с мягким наконечником, которые также расширяются, но в меньшей степени, чем HP.
4. Специальные патроны
Tracer: Патроны с фосфорным наполнителем, позволяющие видеть траекторию полета пули, полезны для корректировки огня.
Subsonic: Патроны, работающие на низкой скорости, что позволяет использовать их с глушителями без значительного звука выстрела.
5. Экспансивные патроны
Fragmenting: Патроны, которые разлетаются на осколки при попадании, нанося урон на большом участке.
Explosive: Патроны с малым зарядом взрывчатого вещества, которые взрываются при попадании, нанося дополнительный урон.
6. Патроны для специальных задач
Non-Lethal: Патроны, предназначенные для подавления без летального исхода (например, резиночки или капсулы с краской).
Incendiary: Патроны, которые воспламеняются при попадании, используемые для создания огненных барьеров или уничтожения техники.
7. Патроны с высокой баллистикой
Match Grade: Высококачественные патроны, предназначенные для соревнований и стрельбы на дальние дистанции, обеспечивающие максимальную точность.
Bonded: Патроны, в которых пуля и сердечник соединены, что увеличивает их целостность и пробивную способность.
8. Патроны с различными типами пороха
Progressive Burning: Патроны с порохом, который сгорает постепенно, обеспечивая более стабильную скорость пули.
Flash Suppressing: Патроны, которые уменьшают вспышку при выстреле, что делает стрельбу менее заметной в темноте.

Кинетическая винтовка.



Огнестрельное оружие
Пистолет с глушителем: Легкий пистолет, идеально подходящий для скрытных операций; снижает звук выстрела и увеличивает точность.
Автоматическая винтовка: Удобное оружие для ближнего и среднего боя с высокой скорострельностью и возможностью модификации.
Снайперская винтовка: Оружие с высокой точностью, позволяющее поражать врагов на большом расстоянии.
Ракетница: Мощное оружие, способное наносить массовый урон и разрушать укрытия.
Плазменное оружие
Плазменный пистолет: Легкое оружие, стреляющее плазменными зарядами, которые наносят урон и могут поджигать врагов.
Плазменная винтовка: Более мощная версия плазменного оружия с возможностью стрельбы очередями и высокой проникающей способностью.
Плазменный гранатомет: Устройство, стреляющее плазменными гранатами, которые взрываются, нанося урон в области.
Плазменный меч: Ближний бой с плазменным лезвием, наносящим высокий урон и способным пробивать броню.
Лазерное оружие
Лазерный пистолет: Компактное оружие, стреляющее концентрированными лазерными лучами, которые наносят точный урон.
Лазерная винтовка: Оружие с высокой скорострельностью, стреляющее лазерными зарядами, способное пробивать укрытия.
Лазерный пулемет: Мощное автоматическое оружие, стреляющее лазерными лучами с высокой скоростью и возможностью перегрева.
Лазерная катана: Ближний бой с лазерным лезвием, которое наносит мгновенный урон, а также может резать через броню.
\subsection{«Искусственный интеллект»}
\subsection{Многопользовательский режим}
\subsection{Интерфейс пользователя}
\subsubsection{Блок-схема}
\subsubsection{Функциональное описание и управление}
\subsubsection{Объекты интерфейса пользователя}
\subsection{Графика и видео}
\subsubsection{Общее описание}
\subsubsection{Двумерная графика и анимация}
\subsubsection{Трехмерная графика и анимация}
\subsubsection{Анимационные вставки}
\subsection{Звуки и музыка}
\subsubsection{Общее описание}
\subsubsection{Звук и звуковые эффекты}
\subsubsection{Музыка}
\subsection{Описание уровней}
\subsubsection{Общее описание дизайна уровней}
\subsubsection{Диаграмма взаимного расположения уровней}
\subsubsection{График введения новых объектов}

\section{Контакты}

\end{document}

